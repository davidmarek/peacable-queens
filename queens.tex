\documentclass[11pt]{article}
\usepackage{cite}
\usepackage{hyperref}
\usepackage{acl2013}
\usepackage[utf8]{inputenc}
\usepackage{times}
\usepackage{url}
\usepackage{latexsym}
\usepackage{todonotes}

\usepackage{amsmath}
\usepackage{tikz}
%\usepackage{algorithm2e}
\usepackage{paralist}
\usepackage{soul}
\usepackage{multirow}
\usepackage{color}
%\usepackage{natbib}

\usepackage{algorithmic}
\usepackage{algorithm}
\def\algorithmautorefname{Algorithm}
\renewcommand*{\sectionautorefname}{Section}

\newcommand{\citet}[1]{\newcite{#1}}
\newcommand{\red}{\color{red}}
\newcommand{\vecs}{\mathbf{s}}
\newcommand{\veco}{\mathbf{o}}
\newcommand{\abs}[1]{\left|#1\right|}

%\setlength\titlebox{6.5cm}    % You can expand the title box if you
% really have to

\title{Peacable Armies of Queens}

\author{David Marek}

\date{}

\begin{document}
\maketitle

\section{Úloha}
V umělé inteligenci je známá úloha $n$-queens, kde je úkolem umístit na
šachovnici $n$ královen tak, aby se navzájem neohrožovali. V úloze Peacable
Armies of Queens (PAoQ) jsou královny rozděleny na dvě skupiny (bílé a černé).
Královny v rámci jedné skupiny se mohou navzájem ohrožovat. Cílem je umístit
$n$ bílých a $n$ černých královen na šachovnici tak, aby žádná bílá
neohrožovala žádnou černou a naopak.

\section{Popis řešení}
Problém PAoQ jsem řešil pomocí splňování omezujících podmínek s využitím
knihovny clpfd. Při řešení jsem experimentoval s více modely, nejprve popíšu
finální model, který úlohu řeší a následně se v další sekci vrátím k dalším
možným modelům a řešením, které jsem zkoušel.

\subsection{Reprezentace pomocí IP}
Výsledné řešení využívá reprezentaci podobnou celočíselnému programování (IP).
Při IP existují pro každé pole šachovnice dvě proměnné:
\[
    b_{ij} = \begin{cases}
        1 & \text{černá královna na poli $(i,j)$} \\
        0 & \text{jinak}
    \end{cases}
\]
\[
    w_{ij} = \begin{cases}
        1 & \text{bílá královna na poli $(i,j)$} \\
        0 & \text{jinak}
    \end{cases}
\]

Maximizujeme 
$$\sum_{i=1}^n\sum_{j=1}^n b_{ij}$$
za podmínek
$$\sum_{i=1}^n\sum_{j=1}^n b_{ij} = \sum_{i=1}^n\sum_{j=1}^n w_{ij}$$
$$b_{ij} + w_{ij} \le 1 ~~~~ \forall ((i_1, j_1), (i_2, j_2)) \in M$$
$$b_{ij}, w_{ij} \in \{0, 1\} ~~~~ \forall 1 \le i, j \le n,$$
kde $M$ je množina párů polí, která sdíli řádek, sloupec anebo diagonálu.

\subsection{Reprezentace pomocí CSP}
Při použití CSP můžeme model použitý pro celočíselné programování zjednodušit.
Každé pole šachovnice je reprezentováno jednou proměnnou:

$$
    s_{ij} = \begin{cases}
        2 & \text{černá královna na poli $(i,j)$}\\
        1 & \text{bílá královna na poli $(i,j)$}\\
        0 & \text{jinak}
    \end{cases}
$$

Podmínku neútočení lze vyjádřit:
$$
\begin{array}[b]{rr}
    & s_{i_1 j_1} = 1 \implies s_{i_2 j_2} \ne 2 \\
    \text{a} & s_{i_1 j_1} = 2 \implies s_{i_2 j_2} \ne 1 \\
             & \forall ((i_1, j_1), (i_2, j_2)) \in M,
\end{array}
$$

kde $M$ je opět množina párů polí, která leží na přímce.

Podmínku lze ještě zjednodušit:
$$ s_{i_1 j_1} + s_{i_2 j_2} \ne 3 ~~~~ \forall ((i_1, j_1), (i_2, j_2)) \in M$$

Tato reprezentace má $N^2$ proměnných a přibližně $4N^3$ binárních podmínek.

\section{Další modely}

Snažil jsem se vytvořit model s menším počtem proměnných. Je možné problém PAoQ
modelovat tak, že proměnné jsou svázány s královnami. Pro každou královnu tak
existovaly dvě proměnné, jedna určující řádek, na kterém královna stojí, a druhá
určující sloupec. Podmínky jsou pak nasnadě. Dvě královny s odlišnou barvou
nesmí mít stejný sloupec, popř. řádek a také nesmí ležet na stejné diagonále.

V tomto případě je ovšem mnohem více symetrií a tak je třeba pro zrychlení
přidat další podmínky. Zabránit permutaci královen lze očíslováním královen a
určením pořadí, v jakém se musí na šachovnici vyskytovat.

Snažil jsem se vytvořit obecný algoritmus pro libovolnou velikost šachovnice,
ale nedařilo se mi zachovat rychlost výpočtu. Při vytváření proměnných pomocí
predikátu {\tt length} došlo k degradaci rychlosti.

\section{Výsledky}

Model vycházející z celočíselného programování je rychlejší pro větší problémy, model s méně parametry je rychlejší pro malé problémy a ty lehčí (když je počet královen menší než optimální počet).

\begin{table}
\begin{center}
\begin{tabular}{|r|r|r|r|}
    \hline
    $N$ & $n$ & Q1      & Q2 \\
    \hline
    \hline
    7   & 7   & 2.2     & 0.7 \\
    \hline
    8   & 8   & 9.7     & 2.5 \\
    \hline
    8   & 9   & 103.0   & 829.5 \\
    \hline
    9   & 9   & 6.7     & 10.0 \\
    \hline
    9   & 10  & 52.7    & 114.2  \\
    \hline
\end{tabular}
\caption{Čas potřebný pro nalezení řešení. $N$ je strana šachovnice, $n$ je
počet královen, Q1 je první model vycházející z IP, Q2 je druhý model s menším
počtem proměnných. Čas je v sekundách.}
\end{center}
\end{table}

\end{document}
